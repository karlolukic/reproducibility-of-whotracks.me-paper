% Options for packages loaded elsewhere
\PassOptionsToPackage{unicode}{hyperref}
\PassOptionsToPackage{hyphens}{url}
%
\documentclass[
]{article}
\usepackage{lmodern}
\usepackage{amssymb,amsmath}
\usepackage{ifxetex,ifluatex}
\ifnum 0\ifxetex 1\fi\ifluatex 1\fi=0 % if pdftex
  \usepackage[T1]{fontenc}
  \usepackage[utf8]{inputenc}
  \usepackage{textcomp} % provide euro and other symbols
\else % if luatex or xetex
  \usepackage{unicode-math}
  \defaultfontfeatures{Scale=MatchLowercase}
  \defaultfontfeatures[\rmfamily]{Ligatures=TeX,Scale=1}
\fi
% Use upquote if available, for straight quotes in verbatim environments
\IfFileExists{upquote.sty}{\usepackage{upquote}}{}
\IfFileExists{microtype.sty}{% use microtype if available
  \usepackage[]{microtype}
  \UseMicrotypeSet[protrusion]{basicmath} % disable protrusion for tt fonts
}{}
\makeatletter
\@ifundefined{KOMAClassName}{% if non-KOMA class
  \IfFileExists{parskip.sty}{%
    \usepackage{parskip}
  }{% else
    \setlength{\parindent}{0pt}
    \setlength{\parskip}{6pt plus 2pt minus 1pt}}
}{% if KOMA class
  \KOMAoptions{parskip=half}}
\makeatother
\usepackage{xcolor}
\IfFileExists{xurl.sty}{\usepackage{xurl}}{} % add URL line breaks if available
\IfFileExists{bookmark.sty}{\usepackage{bookmark}}{\usepackage{hyperref}}
\hypersetup{
  pdftitle={WhoTracks.me Dataset - Variable Descriptions},
  pdfauthor={Karlo Lukic, lukic@wiwi.uni-frankfurt.de},
  hidelinks,
  pdfcreator={LaTeX via pandoc}}
\urlstyle{same} % disable monospaced font for URLs
\usepackage[margin=1in]{geometry}
\usepackage{graphicx}
\makeatletter
\def\maxwidth{\ifdim\Gin@nat@width>\linewidth\linewidth\else\Gin@nat@width\fi}
\def\maxheight{\ifdim\Gin@nat@height>\textheight\textheight\else\Gin@nat@height\fi}
\makeatother
% Scale images if necessary, so that they will not overflow the page
% margins by default, and it is still possible to overwrite the defaults
% using explicit options in \includegraphics[width, height, ...]{}
\setkeys{Gin}{width=\maxwidth,height=\maxheight,keepaspectratio}
% Set default figure placement to htbp
\makeatletter
\def\fps@figure{htbp}
\makeatother
\setlength{\emergencystretch}{3em} % prevent overfull lines
\providecommand{\tightlist}{%
  \setlength{\itemsep}{0pt}\setlength{\parskip}{0pt}}
\setcounter{secnumdepth}{-\maxdimen} % remove section numbering

\title{WhoTracks.me Dataset - Variable Descriptions}
\author{Karlo Lukic,
\href{mailto:lukic@wiwi.uni-frankfurt.de}{\nolinkurl{lukic@wiwi.uni-frankfurt.de}}}
\date{2020-08-03}

\begin{document}
\maketitle

This document represents a summary of WhoTracks.me data compiled from
paper readings, blog posts and official GitHub's variable descriptions.

\hypertarget{data-collection}{%
\section{Data collection}\label{data-collection}}

Data was collected from May 2017 from users that used Cliqz browser
extension. In Feb 2018, 70\% of the data came from German users
according to \href{https://t.ly/NQcD}{this} blog post. Then in March
2018, users of Ghostry FireFox extension -- and Ghostry extension
available for other browsers (Safari, Chrome, Opera and Edge) from users
that opted-in to \emph{HumanWeb} data collection -- were added to the
dataset. This caused a slight decrease in the avg. no. of trackers in
April 2018, since Ghostry users were blocking more trackers. This is
explained in \href{https://bit.ly/2VIBndZ}{this} and
\href{https://t.ly/BoCm}{this} blog posts.

\href{https://t.ly/BoCm}{This} blog post illustrates where the traffic
came from in April 2018: Germany and USA being most representative.

\href{https://bit.ly/2SJG2fj}{This} blog post notes that WhoTracks.me
does not collect data for pages with no trackers; in other words,
collected data for all sites contains some number of third-parties and
tracking.

\hypertarget{datasets}{%
\section{Datasets}\label{datasets}}

There are 5 main datasets on \href{https://bit.ly/39wiim0}{WhoTracks.me
GitHub's repo} -- unlike 4 datasets available in ``Explorer'' section on
the website:

\begin{itemize}
\tightlist
\item
  \texttt{sites.csv}: Stats for number of trackers seen on popular
  websites.
\item
  \texttt{site\_trackers.csv}: Stats for each tracker on each site.
\item
  \texttt{domains.csv}: Top third-party domains seen tracking.
\item
  \texttt{trackers.csv}: Top trackers - this combines domains known be
  operated by the same tracker.
\item
  \texttt{companies.csv}: Top companies - aggregates the stats for
  trackers owned by the same company.
\end{itemize}

\begin{quote}
``We structure each subsection in a way that describes measurements in
the perspective of the parties involved: websites, third parties and
users. This enables us to better measure the dynamics of the industry.''
\end{quote}

In addition, WhoTracks.me team provides up-to-date
\href{https://t.ly/GYfw}{SQL database} that links third-party domains to
trackers, which are then linked to unique companies operating those
trackers. This is similar to \href{https://bit.ly/2ZylQRF}{Disconnect's
Tracker List} and \href{https://t.ly/EHuu}{webXray's Domain Owner List},
yet much more comprehensive. These parties are already categorized
accordingly within the datasets.

\hypertarget{variable-descriptions}{%
\section{Variable descriptions}\label{variable-descriptions}}

All 5 above datasets share similar aggregated variables. The difference
therefore, lies in the \emph{perspective} of each dataset. Here are the
variable descriptions (``contexts'' are added to variables for
groupings):

\textbf{General context}:

\begin{itemize}
\item
  \texttt{site} - one of the most frequently visited websites from a
  proportion of traffic in certain month. That means that the most
  popular websites in the dataset were not generated by Alexa or similar
  services, but by real users. Total number of published
  most-visited-user-generated sites increased over time: e.g.~from 700
  most visited sites in 2017 to over 1000 sites in 2020 in Global
  datasets. Note: average monthly traffic (page loads) of users was
  around 100 million page loads during 2017, and increased to 300-500
  million page loads from April 2018 onward (as described in
  \href{https://t.ly/J68v}{this} part of WhoTracks.me paper). String.
\item
  \texttt{month} - month of observation. Global traffic data starts from
  May 2017 and ends with latest GitHub release; EU/US traffic split
  starts from April 2018 and ends with latest GitHub release. mm-yyyy
  format string/date.
\item
  \texttt{country} - main region where the traffic is coming from:
  e.g.~global, US, EU, DE, FR. String.
\item
  \texttt{category} - site's category (in \texttt{sites.csv}). String.
\item
  \texttt{tracker\_category} - tracker's category (in
  \texttt{sites\_trackers.csv}). Descriptions of tracker categories are
  provided \href{https://bit.ly/2RQ56kk}{here}. String.
\item
  \texttt{popularity} - the relative amount of traffic compared to the
  most popular site (described \href{https://bit.ly/3bpBOAI}{here}).
  Float between 0 and 1.
\end{itemize}

\textbf{Utilised tracking context (stateful)} -- generates more
persistant tracking ID by trackers:

\begin{itemize}
\tightlist
\item
  \texttt{cookies} - proportion of pages where a cookie was sent by the
  browser, or a \texttt{Set-Cookie} header was returned by the tracker's
  server. Float between 0 and 1.
\end{itemize}

\textbf{Utilised tracking context (stateless)} -- generates less
persistant tracking ID by trackers:

\begin{itemize}
\tightlist
\item
  \texttt{bad\_qs} - proportion of pages where a unique identifier (UID)
  was detected in the query string parameters sent with a request to
  this tracker. More on this \href{https://bit.ly/2RWGkyT}{here}. The
  methodology for this detection can be found in
  \href{https://bit.ly/32Qa86H}{the paper}. Float between 0 and 1.
\end{itemize}

\begin{quote}
``Note that these detection methods assume that trackers are not
obfuscating the identifiers they generate.''
\end{quote}

\textbf{Utilised tracking context (stateful + stateless)} -- either
cookie tracking or fingerprinting context, inclusive:

\begin{itemize}
\tightlist
\item
  \texttt{tracked} - proportion of pages where a UID transmission was
  detected, either via \texttt{cookies} or \texttt{bad\_qs}. Float
  between 0 and 1.
\end{itemize}

\begin{quote}
``We define tracking as when a service is able to collect and correlate
data across multiple sites.''
\end{quote}

\newpage

\textbf{Secure context} -- tracker used HTTPS requests -- instead of
HTTP -- to load its content:

\begin{itemize}
\tightlist
\item
  \texttt{https} - proportion of pages where the tracker only used
  \texttt{HTTPS} traffic. Float between 0 and 1.
\end{itemize}

\textbf{Tracking requests context} -- we report the mean number of
third-party requests per page for each tracker, and the subset of these
requests in a tracking context:

\begin{itemize}
\item
  \texttt{requests} - average number of requests made to the tracker per
  page. Positive float.
\item
  \texttt{requests\_tracking} - average number of requests made to the
  tracker with tracking (cookie or query string) per page. Positive
  float.
\end{itemize}

\textbf{Tracking cost context} -- how much page loads do trackers clog
up by being loaded, more on this in the \href{https://t.ly/KnKP}{Tracker
Tax paper}:

\begin{itemize}
\tightlist
\item
  \texttt{content\_length} - average of \texttt{Content-Length} headers
  received per page. This is an approximate measure of the bandwidth
  usage of the tracker. Expressed in bytes. Positive float.
\end{itemize}

\begin{quote}
``As users navigate the web, they load content from websites they visit
as well as the third parties present on the website. \ldots{} Previous
studies have found that each extra third party added to the site will
contribute to an increase of 2.5\% in the site's loading time.''
\end{quote}

\textbf{Tracker's blocking context} -- how often the tracker is affected
by blocklist-based blockers:

\begin{itemize}
\item
  \texttt{requests\_failed} - average number of requests make to the
  tracker per page which do not succeed. In other words, avg. number of
  failed requests per page load (for comparison with \texttt{requests}
  to get an idea of how aggressive the blocking is). This is an
  approximate measure of blocking from external sources (i.e.~adblocking
  extensions or firewalls). Measure \href{https://bit.ly/2RSjBDX}{added}
  in Dec 2017. Positive float.
\item
  \texttt{has\_blocking} - proportion of pages where some kind of
  external blocking of the tracker was detected.Measure
  \href{https://bit.ly/2RSjBDX}{added} in Dec 2017. Float between 0 and
  1.
\end{itemize}

\begin{quote}
``These signals {[}\texttt{requests\_failed} and
\texttt{has\_blocking}{]} should be able to tell us something about the
impact of blocking on different trackers in the ecosystem. For example,
we see evidence of blocking 40\% of the time for Google Analytics and
Facebook {[}in Dec 2017{]}, and between 10\% and 20\% of requests
failing. Thus, anyone using these services to measure activity and
conversions on their sites must reckon with error rates in these orders.
We also can see how new entrants can initially avoid the effects of
blocking - for Tru Optik and Digitrust who we mentioned earlier, we
measure only 5 and 1\% of pages which may be affected by blocking.''
\end{quote}

\newpage

\textbf{Tracker's content loading context} -- proportion of page loads
where specific resource types were loaded by the tracker (e.g.~scripts,
iframes, plugins)

Signals for the frequency with which certain resource types are loaded
by third-parties (measures \href{https://t.ly/NQcD}{added} in Feb 2018):

\begin{itemize}
\tightlist
\item
  \texttt{script} - JavaScript code (via a
  \texttt{\textless{}script\textgreater{}} tag or web worker).
\end{itemize}

\begin{quote}
``If a third-party Javascript file is loaded into the page, the
third-party is given the ability to modify the page at will, intercept
all user input on the page, as well as load any other scripts or third
parties they wish. \ldots{} any third-party which is permitted to load
Javascript in the login document will have to ability to read users'
login information inputted into this page.''
\end{quote}

\begin{itemize}
\tightlist
\item
  \texttt{iframe} - a subdocument (via
  \texttt{\textless{}frame\textgreater{}} or
  \texttt{\textless{}iframe\textgreater{}} elements).
\end{itemize}

\begin{quote}
``Content loaded into an iFrame context is safer, as this is a sandboxed
environment.''
\end{quote}

\begin{itemize}
\tightlist
\item
  \texttt{beacon} - requests sent through the
  \href{https://bit.ly/3hxqbKK}{Beacon API}. More on this
  \href{https://bit.ly/2WPDUEC}{here}:
\end{itemize}

\begin{quote}
``A tracking pixel, is one of various techniques used on web pages or
email, to unobtrusively (usually invisibly) allow checking that a user
has accessed some content.''
\end{quote}

\begin{itemize}
\tightlist
\item
  \texttt{image} - image and imageset resources.
\end{itemize}

\begin{quote}
``Loading third-party images in the page allows the third-party to know
the page you're visiting, via the Referer header, your IP address, and
may allow them to further track your browser via Cookies''
\end{quote}

\begin{itemize}
\item
  \texttt{stylesheet} -
  \href{https://developer.mozilla.org/en-US/docs/Web/CSS}{CSS} files.
\item
  \texttt{font} - custom fonts.
\item
  \texttt{xhr} - requests made from scripts via the
  \href{https://developer.mozilla.org/en-US/docs/Web/API/XMLHttpRequest}{XMLHttpRequest}
  or
  \href{https://developer.mozilla.org/en-US/docs/Web/API/Fetch_API}{fetch}
  APIs.
\item
  \texttt{plugin} - requests of \texttt{object} or
  \texttt{object\_subrequest} types, which are typically associated with
  browser plugins such as Flash.
\item
  \texttt{media} - requests loaded via
  \texttt{\textless{}video\textgreater{}} or
  \texttt{\textless{}audio\textgreater{}} HTML elements.
\end{itemize}

\begin{quote}
``By reporting these {[}above resource types{]} values we can further
characterize tracker behaviours, and quantify risks, such as which
trackers are being permitted to load scripts on certain pages. With this
data we can see that, for example, Google Analytics loads their script
on each page load (98\% of the time), then registers the visit via a
pixel on 59\% of page loads. We also see that on 6\% of pages a request
is also made via the Beacon API. Similarly, if we look at the Webtrekk
tracker, which is present on many popular German websites, we can see
that on sensitive websites such as banking (dkb.de) and health insurance
(tk.de), the tracker is loaded without scripts. This is at least an
indication that in certain contexts website owners are taking care to
minimise the potential risk of a third-party being compromised and
gathering sensitive information from the page, or even collecting
sensitive information by mistake.''
\end{quote}

\textbf{Tracker's presence context} -- there are also counts of
presences of other entities in the aggregation. This enables us to, for
example, count how many of a tracker's domains they use simultaneously
on average, or how many different trackers and companies are usually
present on sites:

\begin{itemize}
\item
  \texttt{hosts} - avg. number of tracker's domains present on site.
  Several domains are grouped under \texttt{trackers}, e.g.:
  \texttt{facebook.com} and \texttt{facebook.net}grouped under
  \texttt{facebook} tracker. Positive float.
\item
  \texttt{trackers} - avg. number of \href{https://t.ly/flZR}{trackers}
  present on site. Trackers are grouped under \texttt{companies}, e.g.:
  \texttt{facebook}, \texttt{facebook\_cdn}, \texttt{facebook\_graph},
  \texttt{...}, grouped under Facebook. Positive float.
\end{itemize}

\begin{quote}
``We define a `tracker' as a third-party domain which is: a) present on
multiple ( \textgreater{} 10 ) different websites with a significant
combined traffic, b) uses cookies or fingerprinting methods in order to
transmit user identifiers''
\end{quote}

\begin{itemize}
\tightlist
\item
  \texttt{companies} - avg. number of companies present on site.
  Positive float.
\end{itemize}

\textbf{Tracker reach context} -- for domain, trackers and companies
aggregations, there are two extra measures:

\begin{itemize}
\tightlist
\item
  \texttt{reach}: Proportional presence across all page loads (i.e.~if a
  tracker is present on 50 out of 1000 page loads, the reach would be
  0.05). Value is a float between 0 and 1.
\end{itemize}

\begin{quote}
``We define a tracker or company's `reach' as the proportion of the web
in which they are included as a third-party.''
\end{quote}

\begin{itemize}
\tightlist
\item
  \texttt{site\_reach}: Presence across unique first party sites.
  e.g.~if a tracker is present on 10 sites, and we have 100 different
  sites in the database, the site reach is 0.1. Value is a float between
  0 and 1.
\end{itemize}

\begin{quote}
``Alternatively, we can measure `site reach', which is the proportion of
websites (unique first-party hostnames) on which this tracker has been
seen at least once.''
\end{quote}

Note: This measure was redefined in Feb 2019 as
\texttt{site\_reach\_top10k}: the number of sites in the top 10,000
which have this tracker on more than 1\% of page loads" according to
\href{https://t.ly/Ra6z}{this} blog post. A further value,
\texttt{site\_avg\_frequency} gives the mean presence across these
sites. Positive floats.

\begin{quote}
``Given that the top 10,000 sites account for 75\% of page loads in our
data, we decided to measure the presence across this fixed set of
sites.''
\end{quote}

Differences between \texttt{reach} and \texttt{site\_reach} (according
to above blog post):

\begin{itemize}
\tightlist
\item
  High reach and high site reach - Ubiquitous presence across both
  popular and less popular sites; A common example of that would be
  Google Analytics.
\item
  High reach and low site reach - Present primarily on few popular,
  high-traffic sites; One such example would be Wikimedia, which, due to
  Wikipedia's popularity, is loaded very often (hence high reach), but
  present on few sites resulting in a low(er) site reach. Another
  example, for similar reasons, would be Ebay Stats,
\item
  Low reach and high site reach - Only appearing rarely on many sites,
  e.g.~only on a small number of pages for each site; In this category
  appear extensions that operate as ``man in the middle'', such as
  Kaspersky Labs.
\item
  Low reach and low site reach - Present on few lower-traffic sites.
  This includes smaller trackers.
\end{itemize}

\textbf{Additional measures without explanation} (todo: contact Sam)

\begin{itemize}
\tightlist
\item
  \texttt{referer\_leaked} - unexplained (my best guess: proportion of
  total page loads in which the HTTP referer header was transmitted to a
  tracker). Float between 0 and 1. todo: double-check this
\end{itemize}

\begin{quote}
``The Referer request header contains the address of the previous web
page from which a link to the currently requested page was followed.''
\end{quote}

\begin{itemize}
\item
  \texttt{referer\_leaked\_header} - unexplained (my best guess:
  proportion of total page loads in which the HTTP referer header was
  transmitted to a tracker while including full URL of the previously
  visited web page). Float between 0 and 1. todo: double-check this
\item
  \texttt{referer\_leaked\_url} - unexplained (no best guess). todo:
  double-check this
\item
  \texttt{cookie\_samesite\_none} - unexplained (no best guess). todo:
  double-check this
\item
  \texttt{t\_active} - unexplained (no best guess). todo: double-check
  this
\end{itemize}

\end{document}
